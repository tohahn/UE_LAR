\documentclass{article}
\usepackage{amsmath}
\usepackage{listings}
\usepackage[utf8]{inputenc}
\usepackage{graphicx}
\usepackage{url}
\usepackage{placeins}

\graphicspath{ {../results/} }

\lstset{
	basicstyle=\footnotesize,
	numbers=left,
	tabsize=3,
	title=\lstname,
	breaklines=true
}

\addtolength{\oddsidemargin}{-.875in}
\addtolength{\evensidemargin}{-.875in}
\addtolength{\textwidth}{1.75in}

\addtolength{\topmargin}{-.875in}
\addtolength{\textheight}{1.75in}

\title{Lernverfahren autonomer Roboter - Übung 4}
\author{Tobias Hahn - 3073375 \\ André Osse - 3066368\\Waldemar Stockmann - 3066\\Markus Strehling - 3066373}	
	
\begin{document}
\maketitle
\newpage
\section*{Übung 4}
\section{K-Means}

\subsection{Implementierung von K-Means}
\lstinputlisting[language=Python]{../code/KMeans.py}

\subsection{Tests von der K-Means-Implementierung}
\lstinputlisting[language=Python]{../code/Test.py}

\subsection{Clustering: Dataset2D}
Wie in der Aufgabe vorgegeben wurde k=3 gewählt. Ebenfalls wurden die initialen Reproduktions-Vektoren der Cluster zufällig gewählt, da mit den ersten k-gewählten Vektoren als Reproduktions-Vektoren keine optimalen aber reproduzierbare Ergebnisse geliefert wurden.

\subsubsection{Erklärung des Plots}
Die roten Punkte repräsentieren die 2D-Vektoren aus dem Datensatz. Das blaue Viereck stellt den Reproduktions-Vektor des jeweiligen Clusters da. Die grünen Vierecke sind die genutzten Vektoren um den Reproduktions-Vektor zu bestimmen.\\
Dabei ist zu beachten, dass diese Vektoren sich ebenfalls in anderen Clustern befinden können, da sie nicht die Reproduktion, sondern die Findung des Reproduktions-Vektors dienen. Desweiteren wenn diese Vektoren zu diesem Zeitpunkt der Iteration weniger optimal liegen, kann es geschehen, dass ein Vektor einem anderen Reproduktions-Vektor zugewiesen wird, als angenommen werden würde. Deshalb kann es geschehen, dass grüne Dreiecke ebenfalls in anderen der 'Haufen' auffindbar wären.\\
In dem Durchlauf, welcher diese Plots darstellen, ist dies jedoch nicht geschehen.

\begin{figure}[h]
    \centering
    \includegraphics[width=\textwidth]{KMeans0.png}
    \caption{Cluster: 1}
    \label{fig:roc}
\end{figure}
\begin{figure}[h]
    \centering
    \includegraphics[width=\textwidth]{KMeans1.png}
    \caption{Cluster: 2}
    \label{fig:roc}
\end{figure}
\begin{figure}[h]
    \centering
    \includegraphics[width=\textwidth]{KMeans2.png}
    \caption{Cluster: 3}
    \label{fig:roc}
\end{figure}

\end{document}
